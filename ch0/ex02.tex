\documentclass[12pt]{article}
\setlength{\parindent}{0in}

\usepackage[margin=1in]{geometry}
\usepackage{amsmath,amsthm,amssymb,amsfonts,braket}

\newcommand{\N}{\mathbb{N}}
\newcommand{\Z}{\mathbb{Z}}
\newtheorem{lemma}{Lemma}

\begin{document}

\title{Exercises for Dummit and Foote Algebra 0.2}
\author{Brandon Kase}
\maketitle

\section*{Exercises}

\subsection*{1}

Computation

\subsection*{2}

Prove that if the integer $k$ divides the integers $a$ and $b$ then $k$ divides $as + bt$ for every pair of integers $s$ and $t$.

\begin{align*}
  k \mid a &\implies a = xk \\
        &\implies as = xks \\
  k \mid b &\implies b = yk \\
        &\implies bt = ykt \\
  as + bt &= xks + ykt \tag*{add the two equations}\\
  as + bt &= k*(xs + yt)
\end{align*}

So $k \mid as + bt$ if we take $r = xs + yt$, since $kr = as + bt$ \qed{}

\subsection*{3}

Prove that if $n$ is composite then there are integers $a$ and $b$ such that $n \mid ab$ but $n \nmid a$ and $n \nmid b$.

By the Fundamental Theorem of Arithmetic. All numbers can be written in the form:

\begin{align*}
  n = p_1^{\alpha_1}p_2^{\alpha_2}\ldots p_s^{\alpha_s}
\end{align*}

Since $n$ is composite, there will be at least two prime factors, let's say $p_a$ and $p_b$, and some other factor $q > 0$:

\begin{align*}
  n &= p_a * p_b * q \\
    &= \underbrace{p_a}_{a} * \overbrace{p_b * q}^{b} \\
    &= ab
\end{align*}

So $n \mid ab$, but $n \nmid a$ and $n \nmid b$ as they are some of the factors of $n$ but not all of the factors $> 1$. \qed{}

\subsection*{4}

Let $a$, $b$ and $N$ be fixed integers with $a \neq 0$ and $b \neq 0$ and let $d = (a, b)$ be the greatest common divisor of $a$ and $b$. Suppose $x_0$ and $y_0$ are particular solutions to $ax + by = N$.
(i.e., $ax_0 + by_0 = N$). Prove for any integer $t$ that the integers

\begin{align*}
  x &= x_0 + \frac{b}{d}t \\
  y &= y_0 - \frac{a}{d}t
\end{align*}

are also solutions to $ax + by = N$.

The following manipulation of terms is entirely $\iff$s so we proceed from the goal down to the assumption.

\begin{align*}
  ax + by &= N \\
  a(x_0 + \frac{b}{d}t) + b(y_0 - \frac{a}{d}t) &= N \\
  ax_0 + \frac{ab}{d}t + by_0 - \frac{ab}{d}t &= N \\
  ax_0 + by_0 &= N
\end{align*} \qed{}

\subsection*{5}

Computation

\subsection*{6}

Prove the Well Ordering Property of $\Z$ by induction and prove the minimal element is unique.

Want to show: If $A$ is any non empty subset of $\Z^+$, $\exists m \in A$ s.t. $m \leq a$, $\forall a \in A$ ($m$ is called a minimal element of $A$).

Order the elements in $A$ by their magnitudes, smallest first. Assuming $|A| = n$:

\begin{align*}
  A = \langle a_0, a_1, \ldots, a_n \rangle
\end{align*}

Want to show $\exists m \in A$, s.t. $\forall a_i \in A$, $m \leq a$. \\

Take $m = a_0$

We proceed by induction over $0 <= i <= n$.

Base case: $i = 0$ \\
$m = a_0 \leq a_0$ \qed{}

IH\@: Assume $m \leq a_i$ \\
Want to show $m \leq a_{i+1}$ \\

Since we ordered the elements in $A$ by their magnitudes, we know $a_i \leq a_{i+1}$.

\begin{align*}
  m &\leq a_i \tag*{by IH} \\
  m &\leq a_i \leq a_{i+1} \tag*{as seen above} \\
  m &\leq a_{i+1}
\end{align*} \qed{}

\subsection*{7}

If $p$ is a prime prove that there do not exist nonzero integers $a$ and $b$ s.t. $a^2 = pb^2$ (i.e., $\sqrt{p}$ is not a rational number).

AFSOC $\sqrt{p}$ is rational \\

\begin{align*}
  \sqrt{p} &= \frac{a}{b} \tag*{by definition of rational} \\
  p &= \frac{a^2}{b^2} \\
  pb^2 &= a^2
\end{align*}

TODO

\end{document}
