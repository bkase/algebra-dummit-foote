\documentclass[12pt]{article}
\setlength{\parindent}{0in}

\usepackage[margin=1in]{geometry} 
\usepackage{amsmath,amsthm,amssymb,amsfonts,braket}
 
\newcommand{\N}{\mathbb{N}}
\newcommand{\Z}{\mathbb{Z}}
\newtheorem{lemma}{Lemma}

\begin{document}
 
\title{Exercises for Dummit and Foote Algebra 0.1}
\author{Brandon Kase}
\maketitle
 
\section*{Exercises}

Let $\mathcal{A}$ be the set set of $2 \times 2$ matrices with real number entries. \\

Let
\begin{align*}
M &= \begin{pmatrix}
1 & 1 \\
0 & 1
\end{pmatrix}
\end{align*} \\

Let
\begin{align*}
\mathcal{B} = \Set{X \in \mathcal{A} | M X = X M}
\end{align*}

\subsection*{1}
Determine which of the following elements of $\mathcal{A}$ lie in $\mathcal{B}$: \\

\begin{align*}
\begin{pmatrix}
1 & 1 \\
0 & 1
\end{pmatrix},
\begin{pmatrix}
0 & 0 \\
0 & 0
\end{pmatrix},
\begin{pmatrix}
1 & 0 \\
0 & 1
\end{pmatrix}
\end{align*}

\subsection*{2}

Prove that if $P,Q \in \mathcal{B}$ then $P + Q \in \mathcal{B}$ (where $+$ denotes the usual sum of two matrices).

\begin{lemma}
$\forall X \in \mathcal{B}$, $X = \begin{pmatrix}p & q \\
0 & p
\end{pmatrix}$
\end{lemma}

Since $X \in \mathcal{B}$, by definition: \\

\begin{align*}
\begin{pmatrix}
1 & 1 \\
0 & 1
\end{pmatrix}
\begin{pmatrix}
p & q \\
r & s
\end{pmatrix} &=
\begin{pmatrix}
p & q \\
r & s
\end{pmatrix}
\begin{pmatrix}
1 & 1 \\
0 & 1
\end{pmatrix} \\
\begin{pmatrix}
p + r & q + s \\
r & s
\end{pmatrix} &=
\begin{pmatrix}
p & p + q \\
r & r + s
\end{pmatrix}
\end{align*}

This induces four equations

\begin{align*}
p + r &= p \\
q + s &= p + q \\
r &= r \\
s &= r + s
\end{align*}

simplifies to

\begin{align*}
r &= 0 \\
s &= p
\end{align*}

Thus all $X \in \mathcal{B}$ are of the form:

\begin{align*}
\begin{pmatrix}
p & q \\
0 & p
\end{pmatrix}
\end{align*}

\qed{}

Want to show if $P,Q \in \mathcal{B}$ then $P + Q \in \mathcal{B}$ (where $+$ denotes the usual sum of two matrices).

\begin{align*}
P &= \begin{pmatrix}
a & b \\
0 & a
\end{pmatrix} \\
Q &= \begin{pmatrix}
x & y \\
0 & x
\end{pmatrix} \\
P + Q &= \begin{pmatrix}
a + x & b + y \\
0 & a + x
\end{pmatrix}
\end{align*}

Take $p = a + x$, $q = b + y$, then: \\

\begin{align*}
P + Q &= \begin{pmatrix}
p & q \\
0 & p
\end{pmatrix} \in \mathcal{B}
\end{align*}

by Lemma 1

\qed{}

\subsection*{3}

Prove that if $P,Q \in \mathcal{B}$ then $P \cdot Q \in \mathcal{B}$ (where $\cdot$ denotes the usual product of two matrices).

\begin{align*}
P &= \begin{pmatrix}
a & b \\
0 & a
\end{pmatrix} \\
Q &= \begin{pmatrix}
x & y \\
0 & x
\end{pmatrix} \\
P \cdot Q &= \begin{pmatrix}
a x & b y \\
0 & a x
\end{pmatrix}
\end{align*}

Take $p = a x$, $q = b y$, then: \\

\begin{align*}
P + Q &= \begin{pmatrix}
p & q \\
0 & p
\end{pmatrix} \in \mathcal{B}
\end{align*}

by Lemma 1

\qed{}

\subsection*{4}

The conditions which determine precisely when $\begin{pmatrix}
p & q \\
r & s
\end{pmatrix} \in \mathcal{B}$ are specified in Lemma 1.

\subsection*{5}

(a) $f : \mathbb{Q} \rightarrow \mathbb{Z}$ defined by $f(\frac{a}{b}) = a$ \\
(b) $f : \mathbb{Q} \rightarrow \mathbb{Q}$ defined by $f(\frac{a}{b}) = \frac{a^2}{b^2}$ \\

Both of these are well-defined because $\forall x \exists! y$ s.t. $f(x) = y$.

\subsection*{6}

$f : \mathbb{R}^{+} \rightarrow \mathbb{Z}$ that maps $r$ to the first digit to the right of the decimal place is well defined as well because for every $r$ there is the same single digit immediately to the right of the decimal place.

\subsection*{7}

Let $f : A \rightarrow B$ be a surjective map of sets. Prove that the relation: $a~b$ if and only if $f(a) = f(b)$ is an equivalence relation whose equivalence classes are the fibers of $f$.

TODO


\end{document}
